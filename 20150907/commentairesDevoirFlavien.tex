\documentclass[a4paper,11pt]{article}
\usepackage{pdflscape}
\usepackage[utf8]{inputenc}
\usepackage[T1]{fontenc}
%\usepackage{fourier} % math & rm
%\usepackage{amsthm,amsfonts,amsmath,amssymb,textcomp}
\usepackage{pst-all,pstricks-add,pst-eucl}
\everymath{\displaystyle}
\usepackage{fp,ifthen}
%\usepackage{color}
%\usepackage{graphicx}
\usepackage{setspace}
\usepackage{array}
\usepackage{tabularx}
\usepackage{supertabular}
\usepackage{hhline}
\usepackage{variations}
\usepackage{enumerate}
\usepackage{pifont}
\usepackage{framed}
\usepackage[fleqn]{amsmath}
\usepackage{amssymb}
\usepackage[framed]{ntheorem}
\usepackage{multicol}
\usepackage{kpfonts}
\usepackage{manfnt}

%\usepackage[hmargin=2.5cm, vmargin=2.5cm]{geometry}
\usepackage{vmargin}          % Pour fixer les marges du document
\setmarginsrb
{1.5cm} 	%marge gauche
{0.5cm} 	  %marge en haut
{1.5cm}     %marge droite
{0.5cm}   %marge en bas
{1cm} 	%hauteur de l'entête
{0.5cm}   %distance entre l'entête et le texte
{1cm} 	  %hauteur du pied de page
{0.5cm}     %distance entre le texte et le pied de page

\newcommand{\R}{\mathbb{R}}
\newcommand{\N}{\mathbb{N}}
%\newcommand{\D}{\mathbb{D}}
\newcommand{\Z}{\mathbb{Z}}
\newcommand{\Q}{\mathbb{Q}}
\newcommand{\C}{\mathbb{C}}
\newcommand{\e}{\text{e}}
\newcommand{\dx}{\text{d}x}
\newcommand{\vect}[1]{\mathchoice%
  {\overrightarrow{\displaystyle\mathstrut#1\,\,}}%
  {\overrightarrow{\textstyle\mathstrut#1\,\,}}%
  {\overrightarrow{\scriptstyle\mathstrut#1\,\,}}%
  {\overrightarrow{\scriptscriptstyle\mathstrut#1\,\,}}}
\newcommand\arraybslash{\let\\\@arraycr}
\renewcommand{\theenumi}{\textbf{\arabic{enumi}}}
\renewcommand{\labelenumi}{\textbf{\theenumi.}}
\renewcommand{\theenumii}{\textbf{\alph{enumii}}}
\renewcommand{\labelenumii}{\textbf{\theenumii.}}

\theoremstyle{break}
\theorembodyfont{\upshape}
\newframedtheorem{Theo}{Théorème}
\newframedtheorem{Prop}{Propriété}
\newframedtheorem{Def}{Définition}

\newtheorem{Rq}{Remarque}
\newtheorem{Ex}{Exemple}
%\newtheorem{exo}{Exercice}

%\theorembodyfont{\small \sffamily}
%\newtheorem{sol}{solution}

\newenvironment{sol}% 
{\def\FrameCommand{\hspace{0.5cm} {\color{black} \vrule width 1pt} \hspace{-0.7cm}}%
  \framed {\advance\hsize-\width}
  \noindent \small \sffamily  %\underline{Solution :}%\\
}%
{\endframed}

\newrgbcolor{vert}{0 0.4 0}
\newrgbcolor{bistre}{1 .50 .30}
\setlength\tabcolsep{1mm}
\renewcommand\arraystretch{1.3}

\everymath{\displaystyle}
\hyphenpenalty 10000 %supprime toutes les césures
%\setcounter{secnumdepth}{0}
%\newcounter{saveenum}

\usepackage[frenchb]{babel}
\usepackage{fancyhdr,lastpage}
\usepackage{fancybox}

%\headheight 15.0 pt
\fancyhead[L]{}
\fancyhead[C]{Commentaires rédaction du début de dm de Flavien.}
\fancyhead[R]{}
\fancyfoot[L]{{\scriptsize\textsl{ Thomas Gire Cité scolaire de Lorgues}}}
%\fancyfoot[C]{\scriptsize\thepage}
%\fancyfoot[C]{\scriptsize\thepage/\pageref{LastPage}}

\title{}
\author{}
\date{}

%\pagestyle{empty}
\pagestyle{fancy}
\usepackage[np]{numprint}

\renewcommand\arraystretch{1.8}

\newcounter{numero}
\newcommand{\exo}{
  \addtocounter{numero}{1}%
  \textbf{\underline{Exercice \arabic{numero}:}}\quad}

\frenchbsetup{StandardEnumerateEnv=true}
\usepackage{etex}
\usepackage{tikz,tkz-tab}

\newframedtheorem{Dev}{Devoirs}
\renewcommand{\theDev}{\empty{}} 

\newcommand{\dm}{
  \textbf{\underline{Devoir à la maison:}}\quad \vspace{0.5cm}}

\begin{document}
  \setlength{\unitlength}{1mm}
  \setlength\parindent{0mm}
  
  
  %\exo
  ~
  \medskip
      
  La formule pour calculer les racines d'un trinôme n'aura pas encore été vue lundi. 
  L'esprit de l'exercice est d'utiliser une autre méthode: ce que l'on verra dans
  d'activité ce lundi. Remarque: On nous demande de déterminer le nombre
  de points d'intersection. Il n'est peut être pas nécessaire de 
  calculer les coordonnées de ces points pour répondre à la question.
  
  \begin{enumerate}
    \item
 L’axe des abscisses est déterminé par f(x)=0 $\to$ pas de sens.

Qui est $f$ ? Cette phrase n'aura pas de sens tant que tu ne préciseras pas ce que 
désigne le symbole $f$. Je constate qu'il n'y a pas de fonction $f$ 
définie dans le contexte.

J'imagine que tu cherches à rappeler une équation de l'axe des abscisses. Très bon réflexe.
Il ne te reste plus qu'à écrire une équation juste pour cet axe. 

Exemple d'équation de droite:

Dire que $y=2x+1$ est une équation pour une droite $D$ signifie que l'équivalence suivante est 
vérifiée:

Un point $M(x;y)$ du plan appartient à $D$ si et seulement si ses coordonnée $x$ et $y$
vérifient l'équation $y=2x+1$.

Conseil de rédaction:

L'axe des abscisses a pour équation ...

Quelle relation est vérifiée par les points sur l'axe des abscisses et seulement par cela ?

Ce que tu voulais certainement dire:

Un point $M(x,y)$ sur la parabole $\mathcal{P}_1:y=f(x)$ où $f(x)=2(x+2)^2+1$ est
sur l'axe des abscisses si et seulement si $f(x)=0$.

D'où l’intérêt de résoudre cette équation.

\item
 On va donc s’appuyer sur cette propriété en faisant un calcul afin de prouver nos 
 affirmations $\to$ Rédaction approximative.
 
 Rédaction alternative: Résolvons donc cette équation.

\item Lorsque l’on a une fonction du type : ax²+bx=c,$\to$ ceci n'est pas une fonction.
Ceci est une équation d'une partie du plan. $g(x)=ax^2+bx$ est une fonction. La fonction constante $h(x)=c$ aussi.

\item on utilise la formule suivante $\frac{-b+\sqrt{b^2-4ac}}{2a}\to$ mais que calcule
cette formule ?

Rédaction alternative : On a la formule ....$=\frac{-b+\sqrt{b^2-4ac}}{2a}$ ou bien
l'expression $\frac{-b+\sqrt{b^2-4ac}}{2a}$ permet de calculer... 

\item La fonction $\mathcal{P}_1$ coupe l’axe des abscisses en 2 points :
$\to \mathcal{P}_1$ n'est pas une fonction. En revanche, $\mathcal{P}_1$ est le graphe
de la fonction $h$ définie par $h(x)=2(x+2)^2+1$ 
(mais il ne coupe pas l'axe des abscisses, ce qu'il faut montrer).

\item
$$P1(x)=0\to h(x)=0$$
$$2(x²+4x+2)+1=0\to 2(x^2+4x+4)+1=0$$.
$$2x²+8x+5=0 \to 2x^2+8x+9 \hspace{1cm} b^2-4ac=8^2-4 \times 2 \times 9<0$$ pas de solution.
$$2x²+8x=-5 \to$$ mauvaise idée de ne pas laisser sous la forme $h(x)=0$
$$(x=(-8+24)/4=-2+264,x=-2-264,S={-2-264;-2+264})$$
(La fonction) $\mathcal{P}_2$ coupe l’axe des abscisses en x points :$\to$ faut il comprendre 
que tu choisis d'appeler $x$ le nombre de points d'intersection ? Ce serait maladroit
car $x$ désigne naturellement une abscisse de points du plan.  
(P2(x)=0)
-x²+6x-7=0
(-x²+6x=7)
$S=\{3-\sqrt{2};3+\sqrt{2}\} \to$ Calcul juste.
  
  \end{enumerate}
\end{document}
