\documentclass[12pt]{article}
\usepackage[T1]{fontenc}
\usepackage[utf8]{inputenc}
\usepackage{fourier}
\usepackage[scaled=0.875]{helvet} 
\renewcommand{\ttdefault}{lmtt} 
\usepackage{amsmath,amssymb,makeidx}
\usepackage[normalem]{ulem}
\usepackage{fancybox,graphicx}
\usepackage{enumerate}
\usepackage{tabularx}
\usepackage{ulem}
\usepackage{dcolumn}
\usepackage{textcomp}
\usepackage{diagbox}
\usepackage{tabularx}
\usepackage{lscape}
\newcommand{\euro}{\eurologo{}}
\usepackage{variations}
%Tapuscrit : Denis Vergès
\usepackage{pstricks,pst-plot,pst-text,pst-tree,pstricks-add}
\setlength\paperheight{297mm}
\setlength\paperwidth{210mm}
\setlength{\textheight}{25cm}
\newcommand{\R}{\mathbb{R}}
\newcommand{\N}{\mathbb{N}}
\newcommand{\D}{\mathbb{D}}
\newcommand{\Z}{\mathbb{Z}}
\newcommand{\Q}{\mathbb{Q}}
\newcommand{\C}{\mathbb{C}}
\renewcommand{\theenumi}{\textbf{\arabic{enumi}}}
\renewcommand{\labelenumi}{\textbf{\theenumi.}}
\renewcommand{\theenumii}{\textbf{\alph{enumii}}}
\renewcommand{\labelenumii}{\textbf{\theenumii.}}
\newcommand{\vect}[1]{\mathchoice%
{\overrightarrow{\displaystyle\mathstrut#1\,\,}}%
{\overrightarrow{\textstyle\mathstrut#1\,\,}}%
{\overrightarrow{\scriptstyle\mathstrut#1\,\,}}%
{\overrightarrow{\scriptscriptstyle\mathstrut#1\,\,}}}
\def\Oij{$\left(\text{O},~\vect{\imath},~\vect{\jmath}\right)$}
\def\Oijk{$\left(\text{O},~\vect{\imath},~\vect{\jmath},~\vect{k}\right)$}
\setlength{\voffset}{-1,5cm}
\usepackage{fancyhdr}
\usepackage{hyperref}
\hypersetup{%
pdfauthor = {APMEP},
pdfsubject = {Baccalauréat S},
pdftitle = {Métropole 22 juin 2015},
allbordercolors = white,
pdfstartview=FitH}   
\thispagestyle{empty}
\usepackage[frenchb]{babel}
\usepackage[np]{numprint}
\begin{document}
\setlength\parindent{0mm}
\marginpar{\rotatebox{90}{\textbf{A. P{}. M. E. P{}.}}}
\rhead{\textbf{A. P{}. M. E. P{}.}}
\lhead{\small Correction du baccalauréat S}
\lfoot{\small{Métropole}}
\rfoot{\small{22 juin 2015}}
\renewcommand \footrulewidth{.2pt}
\pagestyle{fancy}
\thispagestyle{empty}
\begin{center} {\large{\textbf{\decofourleft~Corrigé du baccalauréat S  Métropole 22 juin 2015~\decofourright
}}} 

\end{center}

\vspace{0,5cm}

\textbf{\textsc{Exercice 3 \hfill 5 points}}

\textbf{Candidats n'ayant pas suivi l'enseignement de spécialité} 

\medskip

\begin{enumerate}
\item Résoudre dans l'ensemble $\C$ des nombres complexes l'équation CE) d'inconnue $z$ :
\[z^2 - 8z + 64 = 0.\]
Soit l'équation $z^2 - 8z + 64 = 0$.

\noindent $\Delta=64-4\times 64=-3\times 64<0$.

\noindent L'équation a deux solutions complexes conjuguées :

\noindent $z_1=\dfrac{8+\mathrm{i}\sqrt{3\times 64}}{2}=\boxed{\textcolor{red}{4+4\mathrm{i}\sqrt{3}}}$ et $z_2=\overline{z_1}=\boxed{\textcolor{red}{4-4\mathrm{i}\sqrt{3}}}$.

\end{enumerate}
\end{document}